



\definecolor{darkred}{rgb}{0.26,0.23,0.23}
\definecolor{codegreen}{rgb}{0,0.6,0}
\definecolor{purple}{rgb}{0.65, 0.12, 0.82}

\lstdefinelanguage{JavaScript}{
  keywords={typeof, new, true, false, catch, function, return, null, catch, switch, var, if,  in, while, do, else, case, break},
  keywordstyle=\bfseries,
  ndkeywords={class, export, boolean, throw, implements, import, this},
  ndkeywordstyle=\color{darkgray}\bfseries,
  identifierstyle=\color{black},
  sensitive=false,
  comment=[l]{//},
  morecomment=[s]{/*}{*/},
  commentstyle=\color{purple}\ttfamily,
}

\lstset{ %
  basicstyle=\footnotesize\ttfamily,
  backgroundcolor=\color{white},   % choose the background color; you must add \usepackage{color} or \usepackage{xcolor}; should come as last argument
  breakatwhitespace=false,         % sets if automatic breaks should only happen at whitespace
  breaklines=true,                 % sets automatic line breaking
  captionpos=b,                    % sets the caption-position to bottom
  commentstyle=\color{codegreen},    % comment style
  frame=tb,	                       % adds a frame around the code
  keepspaces=true,                 % keeps spaces in text, useful for keeping indentation of code (possibly needs columns=flexible)
  rulecolor=\color{black},         % if not set, the frame-color may be changed on line-breaks within not-black text (e.g. comments (green here))
  showspaces=false,                % show spaces everywhere adding particular underscores; it overrides 'showstringspaces'
  showstringspaces=false,          % underline spaces within strings only
  showtabs=false,                  % show tabs within strings adding particular underscores
  tabsize=2,	                     % sets default tabsize to 2 spaces
}


%editing

\begin{document}
\setstretch{1.3}
\setlength{\parindent}{1pt}
\noindent


\newpage
\chapter{Appendix}
\section{persp.R}
\begin{lstlisting}[language = R]
## initialize and create a viewport prepare for drawing
perInit = function ( plot, newpage = FALSE, dbox = TRUE ) {
    info = plot
    ## [[1]] is the all the grapical information that transfer into grid
    ## [[3]] is the persp call information
    ## [[2]] is the plot details eg: x, y, z, xlim, ylim, zlim, col ...
    ## create a list that store all information from the persp
    ## then pass the information to per for drawing.
    ## x is [[2]]; y is [[3]]; z is [[4]]
    ## xr is [[5]]; yr is [[6]]; zr is [[7]]
    ## col is [[14]]; border is [[15]]; box is [[19]]
    ## axes is [[20]], nTicks is [[21]]
    ## tickType is [[22]]
    ## xlab/ylab/zlab = [[23]]/[[24]]/[[25]]
    ## main is in plot[[1]][[4]][[2]][[2]]
    ## shade is 0.8, ltheta/lphi = [[16]]/[[17]]
    ## expand is [[13]], scale is [[12]]
    out = list(x = info[[2]], y = info[[3]], z = info[[4]],
                xr = info[[5]], yr = info[[6]], zr = info[[7]],
                col = info[[14]], border = info[[15]][1] ##only allows one color for border
                , dbox = info[[19]],
                newpage = newpage, 
                phi = info[[9]], theta = info[[8]], r = info[[10]], d = info[[11]],
                axes = info[[20]], nTicks = info[[21]], tickType = info[[22]],
                xlab = info[[23]], ylab = info[[24]], zlab = info[[25]],
                ## parameters in 'par' that need added to per
                lwd = info$lwd, lty = info$lty, #col.axis = info$col.axis,
                #col.lab = info$col.lab, 
                cex.lab = info$cex.lab, 
                shade = info[[18]], ltheta = info[[16]], lphi = info[[17]],
                expand = info[[13]], scale = info[[12]]
                #main = plot[[1]][[4]][[2]][[2]]
                )

    if(out$newpage == TRUE)
        grid.newpage()
    out
}

## main call 
C_persp = function(plot = NULL, ...)
{
    dev.set(recordDev())
    par = currentPar(NULL)
    dev.set(playDev())

    #information extraction
    xc = yc = zc = xs = ys = zs = 0
    plot = perInit(plot, newpage = FALSE)
    xr = plot$xr; yr = plot$yr; zr = plot$zr
    xlab = plot$xlab; ylab = plot$ylab; zlab = plot$zlab
    col.axis = plot$col.axis; col.lab = plot$col.lab; 
    col = plot$col; cex.lab = plot$cex.lab
    nTicks = plot$nTicks; tickType = plot$tickType
    expand = plot$expand ;scale = plot$scale
    ltheta = plot$ltheta; lphi = plot$lphi
    main = plot$main; axes = plot$axes
    dbox = plot$dbox; shade = plot$shade
    r = plot$r; d = plot$d; phi = plot$phi; theta = plot$theta

    xs = LimitCheck(xr)[1]
    ys = LimitCheck(yr)[1]
    zs = LimitCheck(zr)[1]
    xc = LimitCheck(xr)[2]
    yc = LimitCheck(yr)[2]
    zc = LimitCheck(zr)[2]

    if(scale == FALSE){
        s = xs
        if(s < ys) s = ys
        if (s < zs) s = zs
        xs = s
        ys = s
        zs = s
    }

    VT = diag(1, 4)
    VT = VT %*% Translate(-xc, -yc, -zc)
    VT = VT %*% Scale(1/xs, 1/ys, expand/zs)
    VT = VT %*% XRotate(-90.0)
    VT = VT %*% YRotate(-theta)
    VT = VT %*% XRotate(phi)
    VT = VT %*% Translate(0.0, 0.0, -r - d)
    trans = VT %*% Perspective(d)

    border = plot$border[1];
    if(is.null(plot$lwd)) lwd = 1 else lwd = plot$lwd
    if(is.null(plot$lty)) lty = 1 else lty = plot$lty
    if(any(!(is.numeric(xr) & is.numeric(yr) & is.numeric(zr)))) stop("invalid limits")
    if(any(!(is.finite(xr) & is.finite(yr) & is.finite(zr)))) stop("invalid limits")


    if(!scale) xs = ys = zs = max(xs, ys, zs)
    colCheck = col2rgb(col, alpha = TRUE)[4,1] == 255
    if(is.finite(ltheta) && is.finite(lphi) && is.finite(shade) && colCheck)
    DoLighting = TRUE else DoLighting = FALSE
    ## check the first color act as Fixcols

    if (DoLighting) Light = SetUpLight(ltheta, lphi)

    # create a viewport inside a 'viewport'
    depth = gotovp(FALSE)
    lim = PerspWindow(xr, yr, zr, trans, 'r')
    #vp = viewport(0.5, 0.5, 1, 1, default.units = 'npc',
    #                xscale = lim[1:2], yscale = lim[3:4])
    upViewport(depth)

    incrementWindowAlpha()
    setWindowPlotAlpha(plotAlpha())
    setUpUsr(lim)

    if (dbox == TRUE) {
        EdgeDone = rep(0, 12)
        if(axes == TRUE){
            depth = gotovp(TRUE)
            PerspAxes(xr, yr, zr, ##x, y, z
                    xlab, ylab, zlab, ## xlab, xenc, ylab, yenc, zlab, zenc
                    nTicks, tickType, trans, ## nTicks, tickType, VT
                    lwd, lty, col.axis, col.lab, cex.lab) ## lwd, lty, col.axis, col.lab, cex.lab
            upViewport(depth)}
    } else {
        EdgeDone = rep(1, 12)
        xr = yr = zr = c(0,0)
    }

    ## draw the behind face first
    ## return the EdgeDone inorder to not drawing the same Edege two times.
    depth = gotovp(TRUE)
    EdgeDone = PerspBox(0, xr, yr, zr, EdgeDone, trans, 1, lwd)
    upViewport(depth)

    depth = gotovp(FALSE)
    DrawFacets(plot = plot, z = plot$z, x = plot$x, y = plot$y,     ## basic
                xs = 1/xs, ys = 1/ys, zs = expand/zs,               ## Light
                col = col,                                          ## cols
                ltheta = ltheta, lphi = lphi, Shade = shade, 
                Light = Light, trans = trans, DoLighting = DoLighting)
    upViewport(depth)

    depth = gotovp(TRUE)
    EdgeDone = PerspBox(1, xr, yr, zr, EdgeDone, trans, 'dotted', lwd)
    upViewport(depth)

}

####Shade function
LimitCheck = function ( lim ) {
    ## not finished yet...
    s = 0.5 * abs(lim[2] - lim[1])
    c = 0.5 * (lim[2] + lim[1])
    c(s, c)
}

XRotate = function ( angle ) {
    TT = diag(1, 4)
    rad = angle * pi / 180
    c = cos(rad)
    s = sin(rad)
    TT[2, 2] = c;
    TT[3, 2] = -s;
    TT[3, 3] = c;
    TT[2, 3] = s;
    TT
}

YRotate = function ( angle ) {
    TT = diag(1, 4)
    rad = angle * pi / 180
    c = cos(rad)
    s = sin(rad)
    TT[1, 1] = c;
    TT[3, 1] = s;
    TT[3, 3] = c;
    TT[1, 3] = -s;
    TT
}

ZRotate = function ( angle ) {
    TT = diag(1, 4)
    rad = angle * pi / 180
    c = cos(rad)
    s = sin(rad)
    TT[1, 1] = c;
    TT[2, 1] = -s;
    TT[2, 2] = c;
    TT[1, 2] = s;
    TT
}

Translate = function(x, y, z)
{
    TT = diag(1,4)
    TT[4, 1] = x
    TT[4, 2] = y
    TT[4, 3] = z
    TT

}

Scale = function(x, y, z)
{
    TT = diag(1,4)    
    TT[1, 1] = x
    TT[2, 2] = y
    TT[3, 3] = z
    TT
}

Perspective = function(d)
{
    TT = diag(1,4)
    TT[3, 4] = -1 / d
    TT
}

        
SetUpLight = function ( theta, phi ) {
    u = c(0, -1, 0, 1)
    VT = diag(1, 4)
    VT = VT %*% XRotate(-phi)
    VT = VT %*% ZRotate(theta)
    Light = u %*% VT
}

FacetShade = function( u, v, Shade, Light ) {
    nx = u[2] * v[3] - u[3] * v[2]
    ny = u[3] * v[1] - u[1] * v[3]
    nz = u[1] * v[2] - u[2] * v[1]
    sum = sqrt(nx * nx + ny * ny + nz * nz)
    if (is.finite(sum)){
        if (sum == 0) sum = 1
        }else{Shade = NA}

    nx = nx/sum
    ny = ny/sum
    nz = nz/sum
    sum = 0.5 * (nx * Light[1] + ny * Light[2] + nz * Light[3] + 1)
    sum^Shade   
}

shadeCol = function( z, x, y, xs, ys, zs, col, ltheta, lphi, Shade, Light) {
    u = v = 0
    shade = 0
    nx = nrow(z)
    ny = ncol(z)
    nx1 = nx - 1
    ny1 = ny - 1
    cols = 0
    ncol = length(col)
    indx = 0:(length(z))
    Light = SetUpLight(ltheta, lphi)
    for(k in 1:(nx1 * ny1)){
        nv = 0
        i = (indx[k]) %% nx1 
        j = (indx[k]) %/% nx1
        icol = (i + j * nx1) %% ncol + 1

        u[1] = xs * (x[i + 2] - x[i + 1])
        u[2] = ys * (y[j + 1] - y[j + 2])
        u[3] = zs * (z[(i + 1)+ j * nx + 1] - z[i + (j + 1) * nx + 1])
        v[1] = xs * (x[i + 2] - x[i + 1])
        v[2] = ys * (y[j + 2] - y[j + 1])
        v[3] = zs * (z[(i + 1) + (j + 1) * nx + 1] - z[i + j * nx + 1])
        icol = (i + j * nx1) %% ncol
        shade[k] = FacetShade(u, v, Shade = Shade, Light = Light)

        shadedCol = col2rgb(col[icol + 1], alpha = TRUE)
        if(is.finite(shade[k])){
            cols[k] = rgb(shade[k] * shadedCol[1], 
                        shade[k] * shadedCol[2], 
                        shade[k] * shadedCol[3], 
                        maxColorValue = 255)
        }else{
            cols[k] = rgb(1,1,1,0)
                        }
    }
        list(cols = cols, shade = shade)
}

## shade end...
PerspBox = function(front = 1, x, y, z, EdgeDone, VT, lty, lwd = lwd )
{
    u0 = u1 = u2 = u3 = 0
    v0 = v1 = v2 = v3 = 0
    for (f in 1:6) {
        p0 = Face[f, 1]
        p1 = Face[f, 2]
        p2 = Face[f, 3]
        p3 = Face[f, 4]

        u0[1] = x[Vertex[p0, 1]]
        u0[2] = y[Vertex[p0, 2]]
        u0[3] = z[Vertex[p0, 3]]
        u0[4] = 1
        u1[1] = x[Vertex[p1, 1]]
        u1[2] = y[Vertex[p1, 2]]
        u1[3] = z[Vertex[p1, 3]]
        u1[4] = 1
        u2[1] = x[Vertex[p2, 1]]
        u2[2] = y[Vertex[p2, 2]]
        u2[3] = z[Vertex[p2, 3]]
        u2[4] = 1
        u3[1] = x[Vertex[p3, 1]]
        u3[2] = y[Vertex[p3, 2]]
        u3[3] = z[Vertex[p3, 3]]
        u3[4] = 1

        v0 = TransVector(u0, VT)
        v1 = TransVector(u1, VT)
        v2 = TransVector(u2, VT)
        v3 = TransVector(u3, VT)
        
        v0 = v0/v0[4]
        v1 = v1/v1[4]
        v2 = v2/v2[4]
        v3 = v3/v3[4]
        
        d = v1 - v0
        e = v2 - v1
        
        nearby = (d[1]*e[2] - d[2]*e[1]) < 0
        
        ## draw the face line by line rather than polygon
        if ((front && nearby) || (!front && !nearby)) {
            if (!EdgeDone[Edge[f, 1]]){
                grid.lines(c(v0[1], v1[1]), c(v0[2], v1[2]), default.units = 'native',
                    gp = gpar(lty = lty, lwd = lwd) 
                    )
                EdgeDone[Edge[f, 1]] = EdgeDone[Edge[f, 1]] + 1
                }
            if (!EdgeDone[Edge[f, 2]]){
                grid.lines(c(v1[1], v2[1]), c(v1[2], v2[2]), default.units = 'native',
                    gp = gpar(lty = lty, lwd = lwd) 
                    )
                EdgeDone[Edge[f, 2]] = EdgeDone[Edge[f, 2]] + 1
                }
            if (!EdgeDone[Edge[f, 3]]){
                grid.lines(c(v2[1], v3[1]), c(v2[2], v3[2]), default.units = 'native',
                    gp = gpar(lty = lty, lwd = lwd) 
                    )
                EdgeDone[Edge[f, 3]] = EdgeDone[Edge[f, 3]] + 1
                }
            if (!EdgeDone[Edge[f, 4]]){
                grid.lines(c(v3[1], v0[1]), c(v3[2], v0[2]), default.units = 'native',
                    gp = gpar(lty = lty, lwd = lwd)
                    )
                EdgeDone[Edge[f, 4]] = EdgeDone[Edge[f, 4]] + 1
                }
        }
    }
    EdgeDone
}

dPolygon = function(x, y, z, col, trans){

    ## the total number of polygon that we need to draw	
    nx = length(x)
    ny = length(y)
    total = nx * ny
    stops = (nx - 1) * (ny - 1)

    ## set the temp value for x,y,z prepare for subsetting
    xTmp = rep(x, length(y))
    yTmp = rep(y,each = nx)
    zTmp = as.numeric(z)

    ## the drawing order is along x-axis, and then along y-axis
    ## then create a vector like a 4Xn matrix, 
    ## i.e the first column contain all the first points for every polygons
    ## the second column contain all the second points for every polygons and so on 
    pBreak = c(1:total, 1 + 1:total, 1 + nx + 1:total, nx + 1:total)
    xBreak = xTmp[pBreak]
    yBreak = yTmp[pBreak]
    zBreak = zTmp[pBreak]

    ## draw the box if required
    ## the vectors now has four paths, every paths contain the information of every points of every polygon
    ## now we need to change the order of this vector, so that the first four index should be the order for drawing 
    ## the first points, not the first four points for the first four polygon
    ## points subsetting 
    plot.index = rep(
        c(1, 1 + total, 
        1 + 2 * total, 
        1 + 3 * total ),
        total) + rep(0:(total - 1), each = 4)

    ## sequence for 'problem's polygons index, e.g
    ## along x-axis, there are n-1 polygons, n is the number of points in x direction
    ## we don't want to draw the nth polygon, hence we deleted those polygon
    dp = rep((4 * seq(nx,total,nx)), each = 4) - (3:0)

    ## final subsetting
    xCoor = xBreak[plot.index][-dp][1 : (4 * stops)]
    yCoor = yBreak[plot.index][-dp][1 : (4 * stops)]
    zCoor = zBreak[plot.index][-dp][1 : (4 * stops)]

    ## vectorize the cols
    colRep = rep_len(col, length(xCoor))

    ## use the first corner of every polygon to determind the order for drawing
    corn.id = 4* 1:(length(xCoor)/4)
    xc = xCoor[corn.id]
    yc = yCoor[corn.id]

    ## method for using the zdepth for changing the drawing order for every polygon
    orderTemp = cbind(xc, yc, 0, 1) %*% trans 
    zdepth = orderTemp[, 4]

    ## the zdepth of a set of 4 points of each polygon
    a = order(zdepth, decreasing = TRUE)
    oo = rep(1:4, length(a)) + rep(a - 1, each = 4) * 4

    xyCoor = trans3d(xCoor[oo],
                    yCoor[oo],
                    zCoor[oo], trans)
                    
    colRep = colRep[a]

    ## record the total number of polygon
    pMax = length(xyCoor$x) / 4
    pout = list(xyCoor = xyCoor, pMax = pMax, 
                colRep = colRep, polygonOrder = a)
    pout
}

DrawFacets = function(plot, z, x, y, xs, ys, zs, 
                        col, ltheta, lphi, Shade,
                        Light, trans, DoLighting)
{
    pout = dPolygon(x, y, z, col, trans)
    xyCoor = pout$xyCoor
    pMax = pout$pMax; colRep = pout$colRep
    polygonOrder = pout$polygonOrder
    polygons = cbind(xyCoor$x, xyCoor$y)
    polygon.id = rep(1:pMax, each = 4)
    col = plot$col

    if (DoLighting == TRUE) {
        col[is.na(col)] = rgb(1, 1, 1)
        if(is.finite(Shade) && Shade <= 0 ) Shade = 1
        shadding = shadeCol(z, x, y,                       ## x, y, z
                xs, ys, zs,                                 ## xs, ys, zs 
                col,                           ## col, ncol
                ltheta, lphi, Shade, Light = Light)         ## ltheta, lphi, Shade(not shade)
        shadedCol = shadding[[1]]
        
        ## clean if any NA's Z-value
        shade = shadding[[2]][polygonOrder]
        misshade = !is.finite(shade)
        misindex = rep(misshade, each = 4)
        polygonOrder = polygonOrder[!misshade]
        polygons = polygons[!misindex,]
        polygon.id = polygon.id[!misindex]
        
        cols = shadedCol[polygonOrder]        
    } else {
        cols = rep_len(col, length(polygons[,1]))[polygonOrder]
    }

    xrange = range(polygons[,1], na.rm = TRUE)
    yrange = range(polygons[,2], na.rm = TRUE)

    grid.polygon(polygons[,1], polygons[,2], id = polygon.id,
                    default.units = 'native', 
                    gp = gpar(col = plot$border, fill = cols, 
                                lty = plot$lty, lwd = plot$lwd)
                    )

}


TransVector = function(u, T) {
    u %*% T
}

lowest = function (y1, y2, y3, y4) {
    (y1 <= y2) && (y1 <= y3) && (y1 <= y4)		
}

labelAngle = function(x1, y1, x2, y2){  

    dx = abs(x2 - x1)
    if ( x2 > x1 ) {
        dy = y2 - y1
    } else {
        dy = y1 - y2
    }

    if (dx == 0) {
        if( dy > 0 ) {
            angle = 90
        } else {
            angle = 270
        }
    } else {
        angle = 180/pi * atan2(dy, dx)
    }
    angle
}	

PerspAxis = function(x, y, z, axis, axisType, 
                    nTicks, tickType, label, 
                    VT, lwd = 1, lty, col.axis = 1,
                    col.lab = 1, cex.lab = 1){

    ## don't know how to use numeric on the switch...
    axisType = as.character(axisType)
    tickType = as.character(tickType)
    u1 = u2 = u3 = c(0.,0.,0.,0.)
    tickLength = .03

    switch(axisType,
           '1' = {min = x[1]; max = x[2]; range = x},
           '2' = {min = y[1]; max = y[2]; range = y},
           '3' = {min = z[1]; max = z[2]; range = z}
            )
            
    d_frac = 0.1 * (max - min)
    nint = nTicks - 1

    if(!nint)nint = nint + 1
    i = nint

    ticks = axisTicks(c(min, max), FALSE, nint = nint)
    min = ticks[1]
    max = ticks[length(ticks)]
    nint = length(ticks) - 1
            
    ## but maybe not this one... haven't test yet...
    while((min < range[1] - d_frac || range[2] + d_frac < max) && i < 20) {
        nint = i + 1
        ticks = axisTicks(c(min, max), FALSE)
        range = range(ticks)
        nint = length(ticks) - 1
    }

    ## axp seems working...
    axp = 0
    axp[1] = min
    axp[2] = max
    axp[3] = nint

    # Do the following calculations for both ticktypes
    # Vertex is a 8*3 matrix; i.e. the vertex of a box
    # AxisStart is a vector of length 8
    # axis is a output 
    # u1, u2 are the vectors in 3-d 
    # the range of x,y,z
    switch (axisType,
        '1' = {
          u1[1] = min
          u1[2] = y[Vertex[AxisStart[axis], 2]]
          u1[3] = z[Vertex[AxisStart[axis], 3]]
        },
        '2' = {
          u1[1] = x[Vertex[AxisStart[axis], 1]]
          u1[2] = min
          u1[3] = z[Vertex[AxisStart[axis], 3]]
        },
        '3' = {
          u1[1] = x[Vertex[AxisStart[axis], 1]]
          u1[2] = y[Vertex[AxisStart[axis], 2]]
          u1[3] = min
        }
    )
    u1[1] = u1[1] + tickLength*(x[2]-x[1])*TickVector[axis, 1]
    u1[2] = u1[2] + tickLength*(y[2]-y[1])*TickVector[axis, 2]
    u1[3] = u1[3] + tickLength*(z[2]-z[1])*TickVector[axis, 3]
    u1[4] = 1

    ##axisType, 1 = 'draw x-axis'
    ##          2 = 'draw y-axis'
    ##          3 = 'draw z-axis'
    switch (axisType,
        '1' = {
        u2[1] = max
        u2[2] = u1[2]
        u2[3] = u1[3]
        },
        '2' = {
        u2[1] = u1[1]
        u2[2] = max
        u2[3] = u1[3]
        },
        '3' = {
        u2[1] = u1[1]
        u2[2] = u1[2]
        u2[3] = max
        }
    )
    u2[4] = 1

    switch(tickType,
        '1' = { 
        u3[1] = u1[1] + tickLength*(x[2]-x[1])*TickVector[axis, 1]
        u3[2] = u1[2] + tickLength*(y[2]-y[1])*TickVector[axis, 2]
        u3[3] = u1[3] + tickLength*(z[2]-z[1])*TickVector[axis, 3]
        },
        '2' = {
        u3[1] = u1[1] + 2.5*tickLength*(x[2]-x[1])*TickVector[axis, 1]
        u3[2] = u1[2] + 2.5*tickLength*(y[2]-y[1])*TickVector[axis, 2]
        u3[3] = u1[3] + 2.5*tickLength*(z[2]-z[1])*TickVector[axis, 3]
        }
    )

    ## u3 is the the labels at the center of each axes
    switch(axisType,
        '1' = {
        u3[1] = (min + max)/2
        },
        '2' = {
        u3[2] = (min + max)/2
        },
        '3' = {
        u3[3] = (min + max)/2
        }
    )
    u3[4] = 1

    ## transform the 3-d into 2-d
    v1 = TransVector(u1, VT)
    v2 = TransVector(u2, VT)
    v3 = TransVector(u3, VT)

    v1 = v1/v1[4]
    v2 = v2/v2[4]
    v3 = v3/v3[4]
      
    ## label at center of each axes
    srt = labelAngle(v1[1], v1[2], v2[1], v2[2])
    #text(v3[1], v3[2], label, 0.5, srt = srt)
    grid.text(label = label, x = v3[1], y = v3[2],
          just = "centre", rot = srt,
          default.units = "native", #vp = 'clipoff',
          gp = gpar(col = col.lab, lwd = lwd, cex = cex.lab)
          )

    ## tickType is not working.. when = '2'
    switch(tickType,
    '1' = {
    arrow = arrow(angle = 10, length = unit(0.1, "in"),
                    ends = "last", type = "open")  
    ## drawing the tick..

    grid.lines(x = c(v1[1], v2[1]), y = c(v1[2], v2[2]),
          default.units = "native", arrow = arrow, #vp = 'clipoff',
          gp = gpar(col = 1, lwd = lwd , lty = lty )
          )
       },
    ## '2' seems working
    '2' = {
        at = axisTicks(range, FALSE, axp, nint = nint)
        lab = format(at, trim = TRUE)
        for(i in 1:length(at)){
            switch(axisType, 
                '1' = {
                u1[1] = at[i]
                u1[2] = y[Vertex[AxisStart[axis], 2]]
                u1[3] = z[Vertex[AxisStart[axis], 3]]
                },
                '2' = {
                u1[1] = x[Vertex[AxisStart[axis], 1]]
                u1[2] = at[i]
                u1[3] = z[Vertex[AxisStart[axis], 3]]
                },
                '3' = {
                u1[1] = x[Vertex[AxisStart[axis], 1]]
                u1[2] = y[Vertex[AxisStart[axis], 2]]
                u1[3] = at[i]
                }
            )
            
            tickLength = 0.03
            
            u1[4] = 1
            u2[1] = u1[1] + tickLength*(x[2]-x[1])*TickVector[axis, 1]
            u2[2] = u1[2] + tickLength*(y[2]-y[1])*TickVector[axis, 2]
            u2[3] = u1[3] + tickLength*(z[2]-z[1])*TickVector[axis, 3]
            u2[4] = 1
            u3[1] = u2[1] + tickLength*(x[2]-x[1])*TickVector[axis, 1]
            u3[2] = u2[2] + tickLength*(y[2]-y[1])*TickVector[axis, 2]
            u3[3] = u2[3] + tickLength*(z[2]-z[1])*TickVector[axis, 3]
            u3[4] = 1
            v1 = TransVector(u1, VT)
            v2 = TransVector(u2, VT)
            v3 = TransVector(u3, VT)
                        
            v1 = v1/v1[4]
            v2 = v2/v2[4]
            v3 = v3/v3[4]
            
            ## Draw tick line
            grid.lines(x = c(v1[1], v2[1]), y = c(v1[2], v2[2]),
                default.units = "native", ##vp = 'clipoff',
                gp = gpar(col = col.axis, lwd = lwd, lty = lty)
                )

            ## Draw tick label
            grid.text(label = lab[i], x = v3[1], y = v3[2],
                just = "centre",
                default.units = "native", #vp = 'clipoff',
                gp = gpar(col = col.axis, adj = 1, pos = 0.5, cex = 1)
                )
            }
        }
    )
}


PerspAxes = function(x, y, z, 
                    xlab, 
                    ylab, 
                    zlab, 
                    nTicks, tickType, VT, 
					## parameters in par
                    lwd = 1, lty = 1, col.axis = 1, col.lab = 1, cex.lab = 1)
{
    xAxis = yAxis = zAxis = 0 ## -Wall 
    u0 = u1 = u2 = u3 = 0

    u0[1] = x[1]; u0[2] = y[1]; u0[3] = z[1]; u0[4] = 1
    u1[1] = x[2]; u1[2] = y[1]; u1[3] = z[1]; u1[4] = 1
    u2[1] = x[1]; u2[2] = y[2]; u2[3] = z[1]; u2[4] = 1
    u3[1] = x[2]; u3[2] = y[2]; u3[3] = z[1]; u3[4] = 1

    v0 = TransVector(u0, VT)
    v1 = TransVector(u1, VT)
    v2 = TransVector(u2, VT)
    v3 = TransVector(u3, VT)

    v0 = v0/v0[4]
    v1 = v1/v1[4]
    v2 = v2/v2[4]
    v3 = v3/v3[4]

    if (lowest(v0[2], v1[2], v2[2], v3[2])) {
        xAxis = 1
        yAxis = 2
    } else if (lowest(v1[2], v0[2], v2[2], v3[2])) {
        xAxis = 1
        yAxis = 4
    } else if (lowest(v2[2], v1[2], v0[2], v3[2])) {
        xAxis = 3
        yAxis = 2
    } else if (lowest(v3[2], v1[2], v2[2], v0[2])) {
        xAxis = 3
        yAxis = 4
    } else
        warning("Axis orientation not calculated")
    ## drawing x and y axes
    PerspAxis(x, y, z, xAxis, '1', nTicks, tickType, xlab, VT, 
                lwd = lwd, lty = lty, col.axis = col.axis, 
                col.lab = col.lab, cex.lab = cex.lab)
                
    PerspAxis(x, y, z, yAxis, '2', nTicks, tickType, ylab, VT, 
                lwd = lwd, lty = lty, col.axis = col.axis, 
                col.lab = col.lab, cex.lab = cex.lab)

    ## Figure out which Z axis to draw
    if (lowest(v0[1], v1[1], v2[1], v3[1])) {
            zAxis = 5
        }else if (lowest(v1[1], v0[1], v2[1], v3[1])) {
            zAxis = 6
        }else if (lowest(v2[1], v1[1], v0[1], v3[1])) {
            zAxis = 7
        }else if (lowest(v3[1], v1[1], v2[1], v0[1])) {
            zAxis = 8
        }else
    warning("Axis orientation not calculated")

    ## drawing the z-axis
    PerspAxis(x, y, z, zAxis, '3', nTicks, tickType, zlab, VT, 
                lwd = lwd, lty = lty, col.axis = col.axis, 
                col.lab = col.lab, cex.lab = cex.lab)
}


PerspWindow = function(xlim, ylim, zlim, VT, style)
{
    xmax = xmin = ymax = ymin = u = 0
    u[4] = 1
    for (i in 1:2) {
        u[1] = xlim[i]
        for (j in 1:2) {
            u[2] = ylim[j]
            for (k in 1:2) {
                u[3] = zlim[k]
                v = TransVector(u, VT)
                xx = v[1] / v[4]
                yy = v[2] / v[4]
                if (xx > xmax) xmax = xx
                if (xx < xmin) xmin = xx
                if (yy > ymax) ymax = yy
                if (yy < ymin) ymin = yy
          }
        }
    }
    pin1 = convertX(unit(1.0, 'npc'), 'inches', valueOnly = TRUE)
    pin2 = convertY(unit(1.0, 'npc'), 'inches', valueOnly = TRUE)
    xdelta = abs(xmax - xmin)
    ydelta = abs(ymax - ymin)
    xscale = pin1 / xdelta
    yscale = pin2 / ydelta
    scale = if(xscale < yscale) xscale else yscale
    xadd = .5 * (pin1 / scale - xdelta);
    yadd = .5 * (pin2 / scale - ydelta);
    ## GScale in C
    xrange = GScale(xmin - xadd, xmax + xadd, style)
    yrange = GScale(ymin - yadd, ymax + yadd, style)
    c(xrange, yrange)

}

GScale = function(min, max, style)
{
  switch(style, 
         'r' = {temp = 0.04 * (max - min)
         min = min - temp
         max = max + temp
         },
         'i' = {}
  )
  c(min, max)
}


## global variables.
TickVector = matrix(ncol = 3, byrow = TRUE, data = c(
    0, -1, -1,
    -1, 0, -1,
    0, 1, -1,
    1, 0, -1,
    -1, -1, 0,
    1, -1, 0,
    -1, 1, 0,
    1, 1, 0 ))

Vertex = matrix(ncol = 3, byrow = TRUE, data = c(
	1, 1, 1,  #xlim[1], ylim[1], zlim[1]
	1, 1, 2,  #xlim[1], ylim[1], zlim[2]
	1, 2, 1,
	1, 2, 2,
	2, 1, 1,
	2, 1, 2,
	2, 2, 1,
	2, 2, 2 ))

Face  = matrix (ncol = 4, byrow = TRUE, data = c(
    1, 2, 6, 5,
    3, 7, 8, 4,
    1, 3, 4, 2,
    5, 6, 8, 7,
    1, 5, 7, 3,
    2, 4, 8, 6 ))

Edge  = matrix (ncol = 4, byrow = TRUE, data = c(
    0, 1, 2, 3,
    4, 5, 6, 7,
    8, 7, 9, 0,
    2,10, 5,11,
    3,11, 4, 8,
    9, 6,10, 1)) + 1
    
AxisStart = c(1, 1, 3, 5, 1, 5, 3, 7)

\end{lstlisting}
\newpage
\section{filled.contour.R}
\begin{lstlisting}[language = R]
## vectorization version  (main in used)
FindPolygonVertices = function(low,  high,
		     x1,  x2,  y1,  y2,
		     z11,  z21,  z12,  z22,
             colrep){

    v1 = FindCutPoints(low, high, x1, y1, x2, y1, z11, z21)
    v2 = FindCutPoints(low, high, y1, x2, y2, x2, z21, z22)
    v3 = FindCutPoints(low, high, x2, y2, x1, y2, z22, z12)
    v4 = FindCutPoints(low, high, y2, x1, y1, x1, z12, z11)

    vx = cbind(v1[[1]], v2[[2]], v3[[1]], v4[[2]])
    vy = cbind(v1[[2]], v2[[1]], v3[[2]], v4[[1]])

    ##  track the coordinate for x and y( if non-NA's)
    index = rowSums(!is.na(vx) )
    ## keep if non-NAs row >= 2 (npt >= 2)
    vx = t(vx)
    vy = t(vy)
    xcoor.na = as.vector(vx[, index > 2])
    ycoor.na = as.vector(vy[, index > 2])
    ## delete all NA's,
    xcoor = xcoor.na[!is.na(xcoor.na)]
    ycoor = ycoor.na[!is.na(ycoor.na)]

    id.length = index[index > 2]
    cols = colrep[index > 2]

    out = list(x = xcoor, y = ycoor, id.length = id.length, cols = cols)
    outs = out
    out
}

FindCutPoints = function(low, high, x1, y1, x2, y2, z1, z2)
{
## inner condiction begin
    ## first ocndiction
    c = (z1 - high) / (z1 - z2)
    cond1 = z1 < high
    cond2 = z1 == Inf
    cond3 = z2 > high | z1 < low

    x.1 = ifelse(cond1, x1, 
              ifelse(cond2, x2, x1 + c * (x2 - x1)))
    x.1 = ifelse(cond3, NA, x.1)
                
    y.1 = ifelse(cond1, y1, 
               ifelse(cond2, y1, y1))
    y.1 = ifelse(cond3, NA, y.1)

    cond4 = z2 == -Inf
    cond5 = z2 <= low
    cond6 = z2 > high | z1 < low

    c = (z2 -low) / (z2 - z1)
    x.2 = ifelse(cond4, x1,
             ifelse(cond5, x2 - c * (x2 - x1), NA))
    x.2 = ifelse(cond6, NA, x.2)
             
    y.2 = ifelse(cond4, y1,
              ifelse(cond5, y1, NA))
    y.2 = ifelse(cond6, NA, y.2)

    ## second condiction
    cond7 = z1 > low
    cond8 = z1 == -Inf
    cond9 = z2 < low | z1 > high

    c = (z1 - low) / (z1 - z2)
    x_1 = ifelse(cond7, x1, 
                ifelse(cond8, x2, x1 + c * (x2 - x1)))
    x_1 = ifelse(cond9, NA, x_1)
                
    y_1 = ifelse(cond7, y1, 
                ifelse(cond8, y1, y1))
    y_1 = ifelse(cond9, NA, y_1)

    cond10 = z2 < high
    cond11 = z2 == Inf
    cond12 = z2 < low | z1 > high
                
    c = (z2 - high) / (z2 - z1)
    x_2 = ifelse(cond10, NA, 
                ifelse(cond11, x1, x2 - c * (x2 - x1)))
    x_2 = ifelse(cond12, NA, x_2)
                
    y_2 = ifelse(cond10, NA, 
                ifelse(cond11, y1, y1))
    y_2 = ifelse(cond12, NA, y_2)
                
    ## third condiction
    cond13 = low <= z1 & z1 <= high
    x..1 = ifelse(cond13, x1, NA)
    y..1 = ifelse(cond13, y1, NA)
## inner condiction end
    
## outer condiction 
    cond14 = z1 > z2
    cond15 = z1 < z2

    xout.1 = ifelse(cond14, x.1,
                ifelse(cond15, x_1,
                        x..1))
    xout.2 = ifelse(cond14, x.2,
                ifelse(cond15, x_2,
                        NA))						

    yout.1 = ifelse(cond14, y.1,
                ifelse(cond15, y_1,
                        y..1))
    yout.2 = ifelse(cond14, y.2,
                ifelse(cond15, y_2,
                        NA))			
## outer condiction end

    ## return x1, x2, y1, y2
    xout = cbind(xout.1, xout.2)
    yout = cbind(yout.1, yout.2)
    list(xout, yout)
}

C_filledcontour = function(plot)
{
    dev.set(recordDev())
    par = currentPar(NULL)
    dev.set(playDev())

    x = plot[[2]]
    y = plot[[3]]
    z = plot[[4]]
    s = plot[[5]]
    cols = plot[[6]]

    ns = length(s)
    nx = length(x)
    ny = length(y)

    x1 = rep(x[-nx], each = ny - 1)
    x2 = rep(x[-1], each = ny - 1)
    y1 = rep(y[-ny], nx - 1)
    y2 = rep(y[-1], nx - 1)

    z11 = as.numeric(t(z[-nx, -ny]))
    z21 = as.numeric(t(z[-1, -ny ]))
    z12 = as.numeric(t(z[-nx, -1]))
    z22 = as.numeric(t(z[-1, -1]))

    x1 = rep(x1, each = ns - 1)
    x2 = rep(x2, each = ns - 1)
    y1 = rep(y1, each = ns - 1)
    y2 = rep(y2, each = ns - 1)
    z11 = rep(z11, each = ns - 1)
    z12 = rep(z12, each = ns - 1)
    z21 = rep(z21, each = ns - 1)
    z22 = rep(z22, each = ns - 1)
    low = rep(s[-ns], (nx - 1) * (ny - 1))
    high = rep(s[-1], (nx - 1) * (ny - 1))

    ## rep color until the same length of x, then subsetting 
    if(length(cols) > ns){
        cols = cols[1:(ns - 1)]
    }else
    {
        cols = rep_len(cols, ns - 1)
    }
    colrep = rep(cols[1:(ns - 1)], nx * ny)
    ## feed color as well as subseeting as x and y
    out = FindPolygonVertices(
                low = low, high = high,
                x1 = x1, x2 = x2, 
                y1 = y1, y2 = y2,
                z11 = z11, z21 = z21, 
                z12 = z12, z22 = z22, colrep = colrep)
    ## actual drawing
    depth = gotovp(TRUE)
    grid.polygon(out$x, out$y, default.units = 'native', id.lengths = out$id.length,
             gp = gpar(fill = out$cols, col = NA))
    upViewport(depth)
}


## for loop version
## identical to C_filledcontour in plot3d.c but very slow
lFindPolygonVertices = function(low,  high,
		     x1,  x2,  y1,  y2,
		     z11,  z21,  z12,  z22,
		     x,  y,  z, npt)
{
    out = list()
    npt = 0
    out1 = lFindCutPoints(low, high, x1,  y1,  z11, x2,  y1,  z21, x, y, z, npt)
    x = out1$x; y = out1$y; z = out1$z; npt = out1$npt

    out2 = lFindCutPoints(low, high, y1,  x2,  z21, y2,  x2,  z22, y, x, z, npt)
    x = out2$x; y = out2$y; z = out2$z; npt = out2$npt

    out3 = lFindCutPoints(low, high, x2,  y2,  z22, x1,  y2,  z12, x, y, z, npt)
    x = out3$x; y = out3$y; z = out3$z; npt = out3$npt
            
    out4 = lFindCutPoints(low, high, y2,  x1,  z12, y1,  x1,  z11, y, x, z, npt)

    out$x = out1$x + out2$y + out3$x + out4$y
    out$y = out1$y + out2$x + out3$y + out4$x
    out$npt = out4$npt
    out
}

lC_filledcontour = function(plot)
{
    dev.set(recordDev())
    par = currentPar(NULL)
    dev.set(playDev())

    x  =  plot[[2]]
    y = plot[[3]]
    z = plot[[4]]
    sc = plot[[5]]
    px = py = pz = numeric(8)
    scol = plot[[6]]

    nx = length(x)
    ny = length(y)
    if (nx < 2 || ny < 2) stop("insufficient 'x' or 'y' values")

    ## do it this way as coerceVector can lose dims, e.g. for a list matrix
    if (nrow(z) != nx || ncol(z) != ny) stop("dimension mismatch")

    nc = length(sc)
    if (nc < 1) warning("no contour values")

    ncol = length(scol)

    depth = gotovp(TRUE)
    for(i in 1:(nx - 1)){
    for(j in 1:(ny - 1)){
        for(k in 1:(nc - 1)){
            npt = 0
            out = lFindPolygonVertices(sc[k], sc[k + 1],
                    x[i], x[i + 1],
                    y[j], y[j + 1],
                    z[i, j],
                    z[i + 1, j],
                    z[i, j + 1],
                    z[i + 1, j + 1],
                    px, py, pz, npt)
            
            npt = out$npt
            
            if(npt > 2)
            { 
                grid.polygon(out$x[1:npt], out$y[1:npt], default.units = 'native',
                    gp = gpar(fill = scol[(k - 1) %% ncol + 1], col = NA), name = 'filled.contour')
            }
        }
    }
    }
    upViewport(depth)

}

lFindCutPoints = function( low,  high,
           x1,  y1,  z1,
           x2,  y2,  z2,
           x,  y,  z,
           npt)
{
    x = y = z = numeric(8)
    if (z1 > z2 ) {
        if (z2 > high || z1 < low){
            return(out = list(x = x, y = y, z = z, npt = npt))
        }

        if (z1 < high) {
            x[npt + 1] = x1
            y[npt + 1] = y1
            z[npt + 1] = z1
            npt = npt + 1
        } else if (z1 == Inf) {
            x[npt + 1] = x2
            y[npt + 1] = y1
            z[npt + 1] = z2
            npt = npt + 1
        } else {
            c = (z1 - high) / (z1 - z2)
            x[npt + 1] = x1 + c * (x2 - x1)
            y[npt + 1] = y1
            z[npt + 1] = z1 + c * (z2 - z1)
            npt = npt + 1
        }
        
        if (z2 == -Inf) {
            x[npt + 1] = x1
            y[npt + 1] = y1
            z[npt + 1] = z1
            npt = npt + 1
        } else if (z2 <= low) {
            c = (z2 -low) / (z2 - z1)
            x[npt + 1] = x2 - c * (x2 - x1)
            y[npt + 1] = y1
            z[npt + 1] = z2 - c * (z2 - z1)
            npt = npt + 1
        }
        
    } else if (z1 < z2) {
        if (z2 < low || z1 > high) {
                return(out = list(x = x, y = y, z = z, npt = npt))
        }
        
        if (z1 > low) {
            x[npt + 1] = x1
            y[npt + 1] = y1
            z[npt + 1] = z1
            npt = npt + 1
        } else if (z1 == -Inf) {
            x[npt + 1] = x2
            y[npt + 1] = y1
            z[npt + 1] = z2
            npt = npt + 1
        } else { 
            c = (z1 - low) / (z1 - z2)
            x[npt + 1] = x1 + c * (x2 - x1)
            y[npt + 1] = y1
                z[npt + 1] = z1 + c * (z2 - z1)
            npt = npt + 1
        }
        
        if (z2 < high) {
        } else if (z2 == Inf) {
            x[npt + 1] = x1
            y[npt + 1] = y1
            z[npt + 1] = z1
            npt = npt + 1
        } else {
            c = (z2 - high) / (z2 - z1)
            x[npt + 1] = x2 - c * (x2 - x1)
            y[npt + 1] = y1
            z[npt + 1] = z2 - c * (z2 - z1)
            npt = npt + 1
        }
    } else {
        if(low <= z1 && z1 <= high) {
            x[npt + 1] = x1
            y[npt + 1] = y1
            z[npt + 1] = z1
            npt = npt + 1
        }
    }
    out = list(x = x, y = y, z = z, npt = npt)
    out
}

\end{lstlisting}
\end{document}
